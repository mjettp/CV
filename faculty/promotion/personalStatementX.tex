\documentclass[12pt,]{article}
\usepackage{lmodern}
\usepackage{amssymb,amsmath}
\usepackage{ifxetex,ifluatex}
\usepackage{fixltx2e} % provides \textsubscript
\ifnum 0\ifxetex 1\fi\ifluatex 1\fi=0 % if pdftex
  \usepackage[T1]{fontenc}
  \usepackage[utf8]{inputenc}
\else % if luatex or xelatex
  \ifxetex
    \usepackage{mathspec}
    \usepackage{xltxtra,xunicode}
  \else
    \usepackage{fontspec}
  \fi
  \defaultfontfeatures{Mapping=tex-text,Scale=MatchLowercase}
  \newcommand{\euro}{€}
    \setmainfont{Georgia}
\fi
% use upquote if available, for straight quotes in verbatim environments
\IfFileExists{upquote.sty}{\usepackage{upquote}}{}
% use microtype if available
\IfFileExists{microtype.sty}{%
\usepackage{microtype}
\UseMicrotypeSet[protrusion]{basicmath} % disable protrusion for tt fonts
}{}
\usepackage[margin=1.0in]{geometry}
\ifxetex
  \usepackage[setpagesize=false, % page size defined by xetex
              unicode=false, % unicode breaks when used with xetex
              xetex]{hyperref}
\else
  \usepackage[unicode=true]{hyperref}
\fi
\hypersetup{breaklinks=true,
            bookmarks=true,
            pdfauthor={},
            pdftitle={},
            colorlinks=true,
            citecolor=blue,
            urlcolor=blue,
            linkcolor=magenta,
            pdfborder={0 0 0}}
\urlstyle{same}  % don't use monospace font for urls
\setlength{\parindent}{0pt}
\setlength{\parskip}{6pt plus 2pt minus 1pt}
\setlength{\emergencystretch}{3em}  % prevent overfull lines
\providecommand{\tightlist}{%
  \setlength{\itemsep}{0pt}\setlength{\parskip}{0pt}}
\setcounter{secnumdepth}{0}

%%% Use protect on footnotes to avoid problems with footnotes in titles
\let\rmarkdownfootnote\footnote%
\def\footnote{\protect\rmarkdownfootnote}

%%% Change title format to be more compact
\usepackage{titling}

% Create subtitle command for use in maketitle
\newcommand{\subtitle}[1]{
  \posttitle{
    \begin{center}\large#1\end{center}
    }
}

\setlength{\droptitle}{-2em}
  \title{}
  \pretitle{\vspace{\droptitle}}
  \posttitle{}
  \author{}
  \preauthor{}\postauthor{}
  \date{}
  \predate{}\postdate{}

\usepackage{booktabs}
\usepackage[font={small},labelfont=bf,labelsep=colon]{caption}
\linespread{1.1}
\usepackage[compact]{titlesec}
\usepackage{enumitem}
\usepackage{tikz}
\def\checkmark{\tikz\fill[scale=0.4](0,.35) -- (.25,0) -- (1,.7) -- (.25,.15) -- cycle;}
\setlist{nolistsep}
\titlespacing{\section}{2pt}{*0}{*0}
\titlespacing{\subsection}{2pt}{*0}{*0}
\titlespacing{\subsubsection}{2pt}{*0}{*0}
\setlength{\parskip}{6pt}

% Redefines (sub)paragraphs to behave more like sections
\ifx\paragraph\undefined\else
\let\oldparagraph\paragraph
\renewcommand{\paragraph}[1]{\oldparagraph{#1}\mbox{}}
\fi
\ifx\subparagraph\undefined\else
\let\oldsubparagraph\subparagraph
\renewcommand{\subparagraph}[1]{\oldsubparagraph{#1}\mbox{}}
\fi

\begin{document}
\maketitle

\pagenumbering{gobble}

\section{Personal Statement}\label{personal-statement}

\emph{Nicholas J. Tustison, DSc}

I am a data scientist specializing in medical image analysis with
technical expertise and international recognition in the development of
high-quality, open-source computational strategies for clinically
oriented research. Ever since returning to the University of Virginia in
2010 as research faculty in the Department of Radiology and Medical
Imaging, I have been the lead technical scientist for various projects
in multiple departments across multiple institutions charged with the
development and deployment of computational techniques for small-,
medium-, and large-scale studies.

Following my undergraduate studies in physics and computer science, I
completed master's and doctoral degrees in biomedical engineering from
the University of Virginia and Washington University in Saint Louis,
respectively, where my work included the development of mathematical
models for quantification of lung and cardiac imaging biomarkers.
Following my graduate training, I continued my post-graduate education
at the University of Pennsylvania under the direction of Dr.~James C.
Gee, a technical pioneer in the area of medical image registration.
While at the University of Pennsylvania, my colleague, Dr.~Brian B.
Avants, and I began development of the Advanced Normalization Tools
(ANTs)---a software package which has since become one of the most
widely used toolkits in the field for imaging data munging and analysis.
Not only is it used by many premier academic research groups, including
those at the University of Virginia, but, due to its open-source nature,
companies such as General Electric (G.E.) and International Business
Machines (I.B.M.) also employ ANTs. These contributions to the field
complement my many other software and algorithmic inclusions in the
Insight Toolkit ---the largest open-source, medical image analysis
software package in the world with origins linked to the Visible Human
Project. In fact, I am one of the top contributors in terms of new
software modules having recently been part of the team which completely
refactored the image registration framework.

Since all too often ``papers are simply advertisements for the
science,'' my colleagues and I have participated in several unbiased
competitions in order to properly evaluate and compare our work. For
example, the ANTs-based Symmetric Normalization (SyN) image registration
framework has been deemed independently to be a top-performing algorithm
for brain, lung, and cardiac image normalization (in addition to being
one of the only algorithms that can be labeled true ``open-source''). In
2013 my colleagues and I won an international competition in Nagoya,
Japan for automatically segmenting brain tumor tissue from multi-modal
MRI. The machine learning technique that we developed for the
competition has since been used by several other groups. Most recently,
in 2015, my colleagues and I developed a complete pipeline for
extracting cortical thickness in the brain which has since been adopted
by several research groups. These measures are extremely salient in
identifying neurodegeneration in diseases such as Alzheimers and other
conditions which affect brain development.

Given my role in the development of such widely used data science
approaches, I have given numerous tutorials at various conferences and
to such recognized groups as the Laboratory of NeuroImaging (LONI) at
the University of Southern California and the Montreal Neurological
Institute associated with McGill University. This has led to an
expansion of my circle of collaborators beyond the University of
Virginia to include a joint appointment at the University of California,
Irvine and a pending appointment at the University of Pennsylvania.

I am honored to be affiliated with the University of Virginia and feel
extremely fortunate to work with its high quality faculty in exploring
interesting research questions. I look forward to pursuing my career
path at UVa where I plan to continue focusing on innovations related to
computational medical image analysis and offering crucial expertise in
for imaging data science.

\end{document}
