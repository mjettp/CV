\documentclass[11pt,]{article}
\usepackage{lmodern}
\usepackage{amssymb,amsmath}
\usepackage{ifxetex,ifluatex}
\usepackage{fixltx2e} % provides \textsubscript
\ifnum 0\ifxetex 1\fi\ifluatex 1\fi=0 % if pdftex
  \usepackage[T1]{fontenc}
  \usepackage[utf8]{inputenc}
\else % if luatex or xelatex
  \ifxetex
    \usepackage{mathspec}
    \usepackage{xltxtra,xunicode}
  \else
    \usepackage{fontspec}
  \fi
  \defaultfontfeatures{Mapping=tex-text,Scale=MatchLowercase}
  \newcommand{\euro}{€}
    \setmainfont{Georgia}
\fi
% use upquote if available, for straight quotes in verbatim environments
\IfFileExists{upquote.sty}{\usepackage{upquote}}{}
% use microtype if available
\IfFileExists{microtype.sty}{%
\usepackage{microtype}
\UseMicrotypeSet[protrusion]{basicmath} % disable protrusion for tt fonts
}{}
\usepackage[margin=1.0in]{geometry}
\ifxetex
  \usepackage[setpagesize=false, % page size defined by xetex
              unicode=false, % unicode breaks when used with xetex
              xetex]{hyperref}
\else
  \usepackage[unicode=true]{hyperref}
\fi
\hypersetup{breaklinks=true,
            bookmarks=true,
            pdfauthor={},
            pdftitle={},
            colorlinks=true,
            citecolor=blue,
            urlcolor=blue,
            linkcolor=magenta,
            pdfborder={0 0 0}}
\urlstyle{same}  % don't use monospace font for urls
\setlength{\parindent}{0pt}
\setlength{\parskip}{6pt plus 2pt minus 1pt}
\setlength{\emergencystretch}{3em}  % prevent overfull lines
\providecommand{\tightlist}{%
  \setlength{\itemsep}{0pt}\setlength{\parskip}{0pt}}
\setcounter{secnumdepth}{0}

%%% Use protect on footnotes to avoid problems with footnotes in titles
\let\rmarkdownfootnote\footnote%
\def\footnote{\protect\rmarkdownfootnote}

%%% Change title format to be more compact
\usepackage{titling}

% Create subtitle command for use in maketitle
\newcommand{\subtitle}[1]{
  \posttitle{
    \begin{center}\large#1\end{center}
    }
}

\setlength{\droptitle}{-2em}
  \title{}
  \pretitle{\vspace{\droptitle}}
  \posttitle{}
  \author{}
  \preauthor{}\postauthor{}
  \date{}
  \predate{}\postdate{}

\usepackage{booktabs}
\usepackage[font={small},labelfont=bf,labelsep=colon]{caption}
\linespread{1.1}
\usepackage[compact]{titlesec}
\usepackage{enumitem}
\usepackage{tikz}
\def\checkmark{\tikz\fill[scale=0.4](0,.35) -- (.25,0) -- (1,.7) -- (.25,.15) -- cycle;}
\setlist{nolistsep}
\titlespacing{\section}{2pt}{*0}{*0}
\titlespacing{\subsection}{2pt}{*0}{*0}
\titlespacing{\subsubsection}{2pt}{*0}{*0}
\setlength{\parskip}{6pt}

% Redefines (sub)paragraphs to behave more like sections
\ifx\paragraph\undefined\else
\let\oldparagraph\paragraph
\renewcommand{\paragraph}[1]{\oldparagraph{#1}\mbox{}}
\fi
\ifx\subparagraph\undefined\else
\let\oldsubparagraph\subparagraph
\renewcommand{\subparagraph}[1]{\oldsubparagraph{#1}\mbox{}}
\fi

\begin{document}
\maketitle

\pagenumbering{gobble}

\section{Research Statement}\label{research-statement}

\emph{Nicholas J. Tustison, DSc}

As research support faculty in the Department of Radiology and Medical
Imaging at UVa, my primary research focus is the development of high
quality and robust software for processing of medical imaging data and
corresponding analysis strategies for these data. To put it simply and
generally, I am a data scientist who operates at the nexus of ``big
data'', statistical science (including machine learning), and software
development for gaining insight into human systems (specifically brain,
lung, and heart).

\subsection{I. Principal Developer and Contributor of The Insight
Toolkit (National Library of Medicine of the National Institutes of
Health)}\label{i.-principal-developer-and-contributor-of-the-insight-toolkit-national-library-of-medicine-of-the-national-institutes-of-health}

Addressing the deficiency of image processing tools for analyzing the
Visible Human Project in 1999, the National Library of Medicine (NLM) of
the National Institutes of Health (NIH) funded the Insight Toolkit (ITK)
initiative bringing together such academic institutions as the
University of Pennsylvania and the University of North Carolina, along
with various industrial partners, such as GE Research. This effort
resulted in the Insight Toolkit---a comprehensive, open-source suite of
implemented algorithms for medical image analysis. Development and
expansion continues to the present and is heavily utilized by industry
and academia worldwide and, due to its generalizability, has been
adopted by the French space agency (CNES) for the processing of
remote-sensing imagery.

\subsubsection{A. Major developer of open source software
contributions}\label{a.-major-developer-of-open-source-software-contributions}

I have provided several key developments to the Insight Toolkit which is
one of the primary venues for my software contributions and influence in
the field. These contributions are listed in the accompanying CV in
Section IX, Subsection C \emph{Open-Source Software Short
Communications}. Starting from my first contribution in 2005
(\emph{\(N\)-D \(C^k\) B-Spline Scattered Data Approximation}), I have
continued to provide open-source algorithmic implementations to the
Insight Toolkit including my latest contribution made recently on August
27, 2016 (\emph{Two Luis Miguel fans walk into a bar in Nagoya
---\textgreater{} (yada, yada, yada) ---\textgreater{} an
ITK-implementation of a popular patch-based denoising filter}). Other
important contributions include operations for image convolution
(\emph{Image Kernel Convolution}), faux Colormapping (\emph{Meeting Andy
Warhol Somewhere Over the Rainbow: RGB Colormapping and ITK}), and
fundamental measures for evaluating segmentation results
(\emph{Introducing Dice, Jaccard, and Other Label Overlap Measures To
ITK}). These software classes have been downloaded over 67,413 times
(average number of downloads per publication = 2,931).

\subsubsection{B. Inventor of the N4 method for MRI bias
correction}\label{b.-inventor-of-the-n4-method-for-mri-bias-correction}

Of all these contributions, perhaps my most significant is a method for
removing the low frequency inhomogeneity artifacts common to MR images
as an important preprocessing step for MR image analysis. This algorithm
is commonly referred to in the literature as ``N4'' or ``Nick's
nonparametric nonuniform intensity normalization'' which is described in
the following publication:

\begin{quote}
Tustison NJ, Avants BB, Cook PA, Egan A, Zheng Y, Yushkevich PA, and Gee
JC. N4ITK: Improved N3 Bias Correction, \emph{IEEE Trans Med Imaging},
29(6):1310--1320, June 2010. Cited 367 times; IF = 3.390; Rank 5 out of
100 computer science, interdisciplinary applications, 12 out of 76
biomedical engineering, 18 out of 249 electrical \& electronic
engineering, 3 out of 24 imaging science \& photographic technology, 21
out of 125 radiology, nuclear medicine \& medical imaging.
\end{quote}

It is a significant extension of the popular N3 algorithm\footnote{Sled
  JG, Zijdenbos AP, and Evans AC. A nonparametric method for automatic
  correction of intensity nonuniformity in MRI data. \emph{IEEE Trans
  Med Imaging}, 17(1):87-97, Feb 1998.} (introduced in 1998 with
currently \textasciitilde{}3,000 citations). Prior to the N4 formal
publication, it was provided as open-source software to the ITK
community:

\begin{quote}
Tustison NJ, Gee JC: N4ITK: Nick's N3 ITK Implementation for MRI Bias
Field Correction, \emph{Insight Journal}, 2009,
\url{http://hdl.handle.net/10380/3053}.
\end{quote}

where it has been downloaded over 10,000 times.

\subsubsection{C. Co-investigator and principal developer of the ITKv4
image registration
refactoring}\label{c.-co-investigator-and-principal-developer-of-the-itkv4-image-registration-refactoring}

Image registration (or the alignment of corresponding features between
two images) is a fundamental component in medical image processing and
analysis. In 2011 the NIH-NLM sponsored a large-scale funding effort to
``modernize'' the Insight Toolkit. One of the three major contracts was
to provide modern image registration techniques requiring a complete
refactoring of the existing image registration framework. This contract
was awarded to a joint team consisting of myself and collaborators from
the University of Pennsylvania (under the direction of Professor James
C. Gee):

\begin{quote}
Sponsor: NIH-NLM\\
Title: \emph{Fundamental Refactoring of Deformable Image Registration in
ITK with Distributed Computing and GPU Acceleration}\\
Role: Principle investigator of UVa subcontract\\
Period: 7/1/2011 -- 6/30/2012
\end{quote}

This team provided several major image registration upgrades to the
algorithmic toolkit where I wrote a significant portion of the actual
software code. Not only did we implement current image registration
technologies for inclusion but we also developed new and innovative
techniques which were also included:

\begin{quote}
Tustison NJ and Avants BB. Explicit B-spline regularization in
diffeomorphic image registration. \emph{Front Neuroinform}, 7:39, 2013.
Cited 21 times; IF = 3.261; Rank 8 out of 57 mathematical \&
computational biology, 105 out of 252 neurosciences.
\end{quote}

\begin{quote}
Avants BB, Tustison NJ, Stauffer M, Song G, Wu B, and Gee JC. The
Insight ToolKit Image Registration Framework. \emph{Front Neuroinform},
8:44, 2014. Cited 29 times; IF = 3.261; Rank 8 out of 57 mathematical \&
computational biology, 105 out of 252 neurosciences.
\end{quote}

\subsubsection{\texorpdfstring{D. Co-investigator and principal
developer of \emph{ITK-Lung: A Software Framework for Lung Image
Processing and
Analysis}}{D. Co-investigator and principal developer of ITK-Lung: A Software Framework for Lung Image Processing and Analysis}}\label{d.-co-investigator-and-principal-developer-of-itk-lung-a-software-framework-for-lung-image-processing-and-analysis}

Consistent with our previous work, Professor James C. Gee and I recently
submitted an NIH R01 grant for the development of ITK-Lung, a set of
open-source software tools for CT, PET, MRI pulmonary image analysis
based on the Insight ToolKit. Specifically, we plan to provide core
algorithms for specific pulmonary image analysis tasks across multiple
modalities, many of which I have included with previous publications.
These basic tasks include intra- and inter-modal pulmonary image
registration, template building for cross-sectional and longitudinal
(i.e., respiratory cycle) analyses, functional and structural lung image
segmentation, perfusion analysis, and computation of quantitative image
indices as potential imaging biomarkers. These efforts would facilitate
other NIH-sponsored projects which interface specific pulmonary
algorithms (e.g., CT nodule detection) with clinical and research
applications. Over the course of this 5-year project, the following UVa
faculty and staff will be engaged:

\begin{itemize}
\tightlist
\item
  Nicholas J. Tustison, DSc, Principal Investigator (50\% / year)
\item
  Kun Qing, PhD, Co-investigator (15\% / year)
\item
  Y. Michael Shim, MD, Co-investigator (2\% / year)
\item
  W. Gerald Teague, MD, Co-investigator (2\% / year)
\end{itemize}

\subsection{II. Co-Founder and Developer of the Advanced Normalization
Tools
(ANTs)}\label{ii.-co-founder-and-developer-of-the-advanced-normalization-tools-ants}

In 2006 my longtime colleague, Dr.~Brian Avants, and I co-founded the
Advanced Normalization Tools (ANTs). ANTs is popularly considered a
state-of-the-art medical image registration and segmentation toolkit
based on ITK. It is used by multiple academic institutions, research
facilities (e.g., the Allen Brain Institute, the Montreal Neurological
Institute, the Laboratory of Neuroimaging at the University of Southern
California), and industry leaders (e.g., IBM Watson, GE Research). In
addition to providing well-performing basic processing components, we
have also engineered advanced pipelines for obtaining key biomarkers for
specific applications.

\subsubsection{A. The ANTs cortical thickness
pipeline}\label{a.-the-ants-cortical-thickness-pipeline}

Measuring the thickness of the cortical gray matter of the brain from
MRI has long been used for assessing various neuropathologies and normal
longitudinal changes in the brain. Up until recently the only publicly
available resource for performing this type of measurement was a
software program called ``FreeSurfer'' which is developed and made
available from Mass General Hospital of Harvard University. Recently,
however, I (along with several colleagues) created an ANTs-based
pipeline which outperformed FreeSurfer on a large, publicly available
data set. This work is described in

\begin{quote}
Tustison NJ, Cook PA, Klein A, Song G, Das SR, Duda JT, Kandel BM, van
Strien N, Stone JR, Gee JC, and Avants BB. Large-Scale Evaluation of
ANTs and FreeSurfer Cortical Thickness Measurements. \emph{NeuroImage},
99:166-179, Oct 2014. Cited 46 times; IF = 6.357; Rank 1 out of 14
neuroimaging, 24 out of 252 neurosciences, 3 out of 125 radiology,
nuclear medicine \& medical imaging.
\end{quote}

All resulting quantities and corresponding scripts and analyses have
been made publicly available for external use. In fact, these
measurements were used recently for investigating other hypotheses
concerning the longitudinal development of the entorhinal cortex:

\begin{quote}
Hasan KM, Mwangi B, Cao B, Keser Z, Tustison NJ, Kochunov P, Frye RE,
Savatic M, and Soares J. Entorhinal cortex thickness across the human
lifespan. J of Neuroimaging, 26(3) :278-82, May 2016. Cited 0 times; IF
= 1.734; Rank 128 out of 192 clinical neurology, 12 out of 14
neuroimaging, and 65 out of 125 radiology, nuclear medicine \& medical
imaging.
\end{quote}

\subsubsection{B. Participant in international medical image analysis
competitions}\label{b.-participant-in-international-medical-image-analysis-competitions}

Over the years our ANTs-based tools have won several international
competitions for a wide variety of applications involving several key
UVa collaborators:

\begin{itemize}
\item
  finished in the first rank in the Klein 2009 international brain
  mapping competition,\footnote{Klein et al., Evaluation of 14 nonlinear
    deformation algorithms applied to human brain MRI registration.
    \emph{NeuroImage}, 46(3):786-802, Jul 2009.}
\item
  finished first overall in the EMPIRE10 international lung mapping
  competition,\footnote{Murphy et al., Evaluation of registration
    methods on thoracic CT: the EMPIRE10 challenge. \emph{IEEE Trans Med
    Imaging}, 30(11):1901-20, Nov 2011.}
\item
  was the standard registration tool for the MICCAI 2013 segmentation
  competitions,\footnote{\url{http://www.miccai2013.org}}
\item
  finished first in the BRATS 2013 challenge,\footnote{\url{http://martinos.org/qtim/miccai2013/}}
  and
\item
  won the best paper award at the STACOM 2014 challenge.\footnote{\url{http://www.springer.com/us/book/9783319146775}}
\end{itemize}

We have provided these winning protocols to the public as open-source
for continued development.

\subsubsection{C. Educator via tutorials and other ANTs informational
fora}\label{c.-educator-via-tutorials-and-other-ants-informational-fora}

I have given several workshops to disseminate a better hands-on
knowledge of the various algorithms and pipelines of the ANTs and ITK
toolkits. These include the following:

\begin{itemize}
\item
  ANTs workshop, MD Anderson, Houston, TX, USA. August 2016.
\item
  ANTs Workshop for the Chronic Effects of Neurotrauma Consortium
  (CENC), Baylor College, Houston, TX, USA. October 2015.
\item
  SimpleITK tutorial, MICCAI, Munich, Germany. October 2015.
\item
  ANTs workshop, Laboratory of Neuroimaging, Marina Del Rey, USA. July
  2015.
\item
  CREATE-MIA Summer Workshop, ANTs Workshop, Montreal, Canada. May 2015.
\item
  SPIE Medical Imaging Workshop, Open source tools for medical image
  analysis, San Diego, USA. February 2012.
\end{itemize}

In addition to these workshops, I respond to several ANTs queries per
week originating from our Sourceforge or Github ANTs repositories. These
inquiries range from instructions for specific programs to providing
analysis guidelines for large-scale studies.

\textbf{Developer for the ANTsR project.} ANTs (Advanced Normalization
Tools) is designed to provide high performance image processing
techniques for medical image analysis. During the evolution of the
toolkit, it became clear that robust statistical machinery was lacking
for making inferences from data produced from ANTs processing and
visualization. As part of a collaborative effort, I am part of the ANTsR
development team which provides an interface between ANTs and the R
project for statistical computing and visualization thus providing a
complete set of tools for multivariate image analysis. ANTsR intends to
provide a modern framework for medical analytics, with a focus on
imaging-assisted prediction and statistical power. The ANTsR package is
publicly available on the github project hosting service

\subsection{III. Research Support}\label{iii.-research-support}

As research faculty, I have played a supportive role for the various
faculty research efforts. These include the following (with
corresponding publications and other items of note):

\subsubsection{A. UVa collaborations}\label{a.-uva-collaborations}

\begin{itemize}
\item
  As part of the Hyperpolarized Gas group at UVa, I support efforts for
  quantitative assessment of functional lung imaging using
  hyperpolarized gases. Collaborators include Talissa Altes (now at
  University of Missouri, Columbia), John Mugler, Eduard de Lange, Kun
  Qing, Jaime Mata, Lucia Flors-Basco, W. Gerald Teague, and Mike Shim.

  \begin{itemize}
  \tightlist
  \item
    These collaborative efforts have resulted in several publications
    including the following:
  \end{itemize}

  \begin{quote}
  Flors L, Mugler JP, De Lange EE, Miller GW, Mata JF, Tustison N, Ruset
  IC, Hersman WW, and Altes TA. Hyperpolarized Gas Magnetic Resonance
  Lung Imaging in Children and Young Adults, \emph{J Thorac Imag},
  31(5):285-295, Sep 2016. Cited 0 times; IF = 1.723; Rank 71 out of 124
  radiology, nuclear medicine, and medical imaging.
  \end{quote}

  \begin{quote}
  Tustison NJ, Qing K, Wang C, Altes TA, and Mugler III JP. Atlas-based
  estimation of lung and lobar anatomy in proton MRI. \emph{Magn Reson
  Med}, 76(1):315-20, Jul 2016. Cited 1 times; IF = 3.571; Rank 20 out
  of 125 radiology, nuclear medicine \& medical imaging
  \end{quote}

  \begin{quote}
  Altes TA, Mugler JP, III, Ruppert K, Tustison NJ, Gersbach J,
  Szentpetery S, Meyer CH, de Lange EE, and Teague WG. Clinical
  Correlates of Lung Ventilation in Asthmatic Children. \emph{J Allergy
  Clin Immun}, 137(3) :789-796, Mar 2016. Cited 2 times; IF = 11.476;
  Rank 1 out of 24 allergy, 6 out of 148 immunology.
  \end{quote}

  \begin{quote}
  Qing K, Altes TA, Tustison NJ, Feng X, Chen X, Mata JF, Miller GW, de
  Lange EE, Tobias WA, Cates GD, Jr., Brookeman JR, and Mugler JP, III.
  Rapid Acquisition of Helium-3 and Proton 3D Image Sets of the Human
  Lung in a Single Breath-hold using Compressed Sensing. \emph{Magn
  Reson Med}, 74(4):1110-5, October 2015. Cited 3 time; IF = 3.571; Rank
  20 out of 125 radiology, nuclear medicine \& medical imaging.
  \end{quote}

  \begin{quote}
  Teague WG, Tustison NJ, and Altes TA. Ventilation Heterogeneity in
  Asthma. \emph{J Asthma}, 51(7):677-84, Sept 2014. Cited 7 times; IF =
  1.854; Rank 18 out of 25 allergy, 39 out of 58 respiratory system.
  \end{quote}

  \begin{quote}
  Tustison NJ*, Avants BB, Altes TA, de Lange EE, Mugler III JP, and Gee
  JC. Ventilation-Based Segmentation of the Lungs Using Hyperpolarized
  3He MRI, \emph{J Magn Reson Imaging}, 34(4):831--841, October 2011.
  Cited 26 times; IF = 3.210; Rank 23 out of 125 radiology, nuclear
  medicine and medical imaging.
  \end{quote}

  \begin{quote}
  Tustison NJ, Altes TA, Song G, de Lange EE, Mugler III JP, and Gee JC.
  Feature Analysis of Hyperpolarized Helium-3 Pulmonary MRI: A Study of
  Asthmatics versus Non-Asthmatics, \emph{Magn Reson Med},
  63(6):1448--1455, June 2010. Cited 31 times; IF = 3.571; Rank 20 out
  of 125 radiology, nuclear medicine \& medical imaging.
  \end{quote}

  \begin{itemize}
  \tightlist
  \item
    Past funding sources:
  \end{itemize}

  \begin{quote}
  Sponsor: NIH-NHLBI\\
  Title: Simultaneous Xe129 MRI of Regional Lung Ventilation and Gas
  Uptake in COPD\\
  Role: Co-investigator\\
  Period: 7/1/2011 -- 5/31/2016
  \end{quote}

  \begin{quote}
  Sponsor: NIH-NHLBI\\
  Title: Single-session bronchial thermoplasty for severe asthmatics
  guided by Hxe MRI Role: Principal investigator on UVa subcontract\\
  Period: 9/1/2011 -- 1/31/2015
  \end{quote}

  \begin{quote}
  Sponsor: NIH-NHLBI\\
  Title: Regulatory Advancement of HXe as an MRI Contrast Agent\\
  Role: Co-investigator\\
  Period: 9/1/2011 -- 1/31/2015
  \end{quote}

  \begin{quote}
  Sponsor: Novartis Pharmaceuticals Corp.\\
  Title: Hyperpolarized noble-gas enhanced imaging of b2-agonist
  pharmacodynamics and pharmacokinetics in mild to moderate asthma\\
  Role: Co-investigator\\
  Period: 10/15/2010 -- 5/31/2014
  \end{quote}

  \begin{quote}
  Sponsor: Vertex Pharmaceuticals, Inc.\\
  Title: A Phase II, Single-Blind, Placebo-Controlled Crossover Study to
  Evaluate the Effect of VX-770 on Hyperpolarized Helium-3 Magnetic
  Resonance Imaging in Subjects with Cystic Fibrosis, the G551D Mutation
  and FEV1 ≥40\% Predicted\\
  Role: Physicist\\
  Period: 9/9/2010 -- 9/8/2012
  \end{quote}

  \begin{itemize}
  \item
    In order to support ongoing software development efforts associated
    with these collaborations and other external collaborations at the
    University of Pennsylvania (cf Subsection D. External
    collaborations), my colleague Dr.~James C. Gee and I recently (Oct.
    2016) submitted an R01 grant titled \emph{ITK-Lung: A Software
    Framework for Lung Image Processing and Analysis} (cf Section I D).
  \item
    Other current and pending grants:
  \end{itemize}

  \begin{quote}
  Title: Hyperpolarized Xenon-129 MRI: a new multi-dimensional biomarker
  to determine pulmonary physiologic responses to COPD therapeutics\\
  Role: Co-investigator\\
  Period: 5 years
  \end{quote}

  \begin{quote}
  Sponsor: NIH-NHLBI\\
  Title: Xe129 MRI of the lung: A new technology to assess treatment for
  COPD\\
  Role: Co-investigator\\
  Period: 7/1/2016 -- 6/30/2017
  \end{quote}
\item
  One of my principal collaborators is James Stone with whom I have been
  developing quantitative methods for traumatic brain injury although
  much of our work has been of much more general neuorimaging
  application.

  \begin{itemize}
  \tightlist
  \item
    These collaborative efforts have resulted in several publications
    including the following:
  \end{itemize}

  \begin{quote}
  Stone JR, Wilde EA, Taylor BA, Tate DF, Levin H, Bigler ED, Scheibel
  RS, Newsome MR, Mayer AR, Abildskov T, Black GM, Lennon MJ, York GE,
  Agarwal R, DeVillasante J, Ritter JL, Walker PB, Ahlers ST, and
  Tustison NJ. Supervised learning technique for the automated
  identification of white matter hyperintensities in traumatic brain
  injury, \emph{Brain Inj}, In press. Cited 0 times; IF = 1.822; Rank
  187 out of 256 neurosciences and 17 out of 65 rehabilitation.
  \end{quote}

  \begin{quote}
  Wilde EA, Bigler ED, Huff TJ, Wang H, Black GM, Christensen Z,
  Goodrich-Hunsaker N, Petrie JA, Abildskov T, Taylor BA, Stone JR,
  Tustison NJ, Newsome MR, Levin HS, Chu ZD, York GE, and Tate DF.
  Quantitative Structural Neuroimaging of Mild Traumatic Brain Injury in
  the Chronic Effects of Neurotrauma Consortium (CENC): Comparison of
  Volumetric Data within and across Scanners, \emph{Brain Inj}, In
  press. Cited 0 times; IF = 1.822; Rank 187 out of 256 neurosciences
  and 17 out of 65 rehabilitation.
  \end{quote}

  \begin{quote}
  Tustison NJ, Cook PA, Klein A, Song G, Das SR, Duda JT, Kandel BM, van
  Strien N, Stone JR, Gee JC, and Avants BB. Large-Scale Evaluation of
  ANTs and FreeSurfer Cortical Thickness Measurements.
  \emph{NeuroImage}, 99:166-179, Oct 2014. Cited 46 times; IF = 6.357;
  Rank 1 out of 14 neuroimaging, 24 out of 252 neurosciences, 3 out of
  125 radiology, nuclear medicine \& medical imaging.
  \end{quote}

  \begin{quote}
  Tustison NJ, Avants BB, Cook PA, Kim J, Whyte J, Gee JC, and Stone JR.
  Logical Circularity in voxel-based analysis: normalization strategy
  may induce statistical bias. \emph{Hum Brain Mapp}, 35:745-759, March
  2014. Cited 21 times; IF = 5.969; Rank 2 out of 14 neuroimaging, 27
  out of 252 neurosciences, 5 out of 125 radiology, nuclear medicine \&
  medical imaging.
  \end{quote}

  \begin{itemize}
  \tightlist
  \item
    Past funding sources:
  \end{itemize}

  \begin{quote}
  Sponsor: The Geneva Foundation\\
  Title: Brain Injury Biomarkers and Behavioral Characterization of mTBI
  in Soldiers Following Repeated, Low-Level Blast Exposure\\
  Role: Co-investigator\\
  Period: 1/1/2013 -- 5/31/2015
  \end{quote}

  \begin{quote}
  Sponsor: Naval Medical Research Center\\
  Title: Experienced Breacher Study\\
  Role: Co-investigator -- UVa subcontract\\
  Period: 6/1/2012 -- 5/30/2014
  \end{quote}

  \begin{itemize}
  \tightlist
  \item
    Current and pending funding sources:
  \end{itemize}

  \begin{quote}
  Sponsor: NASA/Medical University of South Carolina\\
  Title: Human Cerebral Vascular Autoregulation and Venous Outflow In
  Response to Microgravity-Induced Cephalad Fluid Redistribution\\
  Role: Co-investigator\\
  Period: 5/16/2013 -- 5/15/2018
  \end{quote}

  \begin{quote}
  Member Chronic Effects of Neurotrauma Consortium\\
  Funding period: 2/2016 - 8/2018
  \end{quote}
\item
  Other successful collaborations with UVa faculty include:

  \begin{itemize}
  \tightlist
  \item
    Automatic segmentation of brain tumor from multi-modal MRI with
    collaborators Max Wintermark (now at Stanford) and former radiology
    fellow Christopher Durst. Our team won the international 2013
    Multimodal Brain Tumor Segmentation Challenge (BRATS) which resulted
    in the following publications:
  \end{itemize}

  \begin{quote}
  Tustison NJ, Shrinhidi KL, Wintermark M, Durst CR, Kandel BM, Gee JC,
  Grossman MC, and Avants BB. Optimal symmetric multimodal templates and
  concatenated random forests for supervised brain tumor segmentation
  (simplified) with ANTsR. Neuroinformatics, 13(2):209-225, April 2015.
  Cited 17 times; IF =2.825; Rank 13 out of 102 computer science,
  interdisciplinary applications, 124 out of 252 neurosciences.
  \end{quote}

  \begin{quote}
  Menze BH, Jakab A, Bauer S, Kalpathy-Cramer J, Farahani K, Kirby J,
  Burren Y, Porz N, Slotboom J, Wiest R, Lanczi L, Gerstner E, Weber
  M-A, Arbel T, Avants BB, Ayache N, Buendia P, Collins DL, Cordier N,
  Corso JJ, Criminisi A, Das T, Delingete H, Demiralp C, Durst CR, Dojat
  M, Doyle S, Festa J, Forbes F, Geremia E, Glocker B, Golland P, Guo X,
  Hamamci A, Iftekharuddin KM, Jena R, John NM, Konukoglu E, Lashkari D,
  Mariz JA, Meier R, Pereira S, Precup D, Price SJ, Riklin-Raviv T, Reza
  SMS, Ryan M, Schwartz L, Shin H-C, Shotton J, Silva CA, Sousa N,
  Subbanna NK, Szekely G, Taylor TJ, Thomas OM, Tustison NJ, Unal G,
  Vasseur F, Wintermark M, Ye DH, Zhao L, Zhao B, Zikic D, Prastawa M,
  Reyes M, and Leemput KV. The Multimodal Brain Tumor Image Segmentation
  Benchmark (BRATS). IEEE Trans Med Imaging, 34(10):1993-2024, October
  2015. Cited 131 times; IF = 3.390; Rank 5 out of 100 computer science,
  interdisciplinary applications, 12 out of 76 biomedical engineering,
  18 out of 249 electrical \& electronic engineering, 3 out of 24
  imaging science \& photographic technology, 21 out of 125 radiology,
  nuclear medicine \& medical imaging.
  \end{quote}

  \begin{itemize}
  \item
    ANTs image registration techniques are currently used as part of
    Mike Salerno's (Cardiac, Radiology \& Medical Imaging, and
    Biomedical Engineering Departments) ongoing development of novel
    cardiac MRI acquisition techniques. As part of this collaborative
    work, Mike and I won the best paper award at the Statistical Atlases
    and Computational Modelling of the Heart (STACOM) motion estimation
    challenge held in Boston, 2014 as part of the international Medical
    Image Computing and Computer-Assisted Intervention (MICCAI)
    conference.
  \item
    Recently, Tony Filiano and Jonathon Kipnis of the Neuroscience
    Department of UVa published a breakthrough paper in Nature
    concerning the relationship between immunity and social deficits. I
    was instrumental in that publication in performing the resting state
    fMRI analysis using some of the software I developed within the
    ANTsR package:
  \end{itemize}

  \begin{quote}
  Filiano AJ, Xu Y, Tustison NJ, Marsh RL, Baker W, Smirnov I, Overall
  CC, Gadani SP, Turner SD, Weng Z, Peerzade SN, Chen H, Lee KS, Scott
  MM, Beenhakker MP, Litvak V, and Kipnis J*. Unexpected role of
  interferon-γ in regulating neuronal connectivity and social behaviour,
  Nature, 535(7612):425-9, Jul 2016. Cited 3 times; IF = 38.138; Rank 1
  out of 63 multidisciplinary sciences.
  \end{quote}

  \begin{itemize}
  \tightlist
  \item
    Other current research collaborators include Spencer Payne and Larry
    Borish (Medicine), Carlos Leiva Salinas (Radiology and Medical
    Imaging), and Stuart Berr (Radiology \& Medical Imaging and
    Biomedical Engineering).
  \end{itemize}
\end{itemize}

\subsubsection{B. External
collaborations}\label{b.-external-collaborations}

\begin{itemize}
\item
  Jim Gee is an Associate Professor in the Department of Radiology at
  the University of Pennsylvania and the Director of the Penn Image
  Computing and Science Laboratory. He was also my post-doc mentor from
  2004 -- 2010. We continue to work together on multiple projects
  including the ITK Lung grant previously described. Short-term future
  plans include a planned ANTs software maintenance R01 grant to be
  submitted in the February 2017 cycle.
\item
  Mike Yassa is an Associate Professor in the Department of Neurobiology
  and Behavior and the Director of the Center for the Neurobiology of
  Learning \& Member. He runs the Yassa Translational Neurobiology Lab
  where I currently have a joint appointment with UC Irvine as a
  Visiting Assistant Researcher. As a long-time user and promoter of the
  ANTs software, Mike and I have several shared projects related to
  Alzheimer's disease and other neurobiological research questions.
\end{itemize}

\end{document}
